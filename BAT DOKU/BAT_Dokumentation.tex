%%%%%%% EINLEITUNG %%%%%%%
% Das Lastenheft beschreibt das Grobkonzept eines Projektes aus sicht des Auftraggebers.
% Es beschreibt lediglich die Basisanforderungen und geht nicht tief ins Detail.
%
% In dem Pflichtenheft wird daher vom Projektbearbeiter dargelegt, 
% wie er die Aufgabenstellung interpretiert und in welchen Schritten er im Detail die Aufgabe abarbeiten m�chte.
% Das Pflichtenheft muss mit dem Auftraggeber (Dozent) abgestimmt werden und
% gilt danach als verbindliche Grundlage f�r die Bewertung der in dem Projekt erbrachten Leistungen.
%

%%%%%%% PR�AMBEL %%%%%%%
% Dokumentenklasse% Dokumenteneinstellungen
\documentclass[
a4paper,
%landscape,
%twocolumn,
twoside=false,
%BCOR=10mm, %offset f�r Binden
hedainclude,
footinclude,
fontsize=11pt,
pagesize=auto,
parskip=half,
headsepline=true,
footsepline = true,
plainheadsepline = true,
plainfootsepline = true,
numbers=noenddot, % enddot, noenddot
draft=false]
{scrbook}

%% List of equation
 %%gmedina solution
  

% Stylesheet definition laden
\usepackage{pkg/layout}
\newcommand\tab[1][1cm]{\hspace*{#1}}
\renewcommand*{\acsfont}[1]{\textnormal{#1}}



\newcommand{\listequationsname}{Formeln}
\newlistof{myequations}{equ}{\listequationsname}
\newcommand{\myequations}[1]{%
	\addcontentsline{equ}{myequations}{\hspace{0.1cm}\protect\numberline{\theequation}#1}\par}





% Beginn des Dokuments
\begin{document}

	% Dokumentsprache
	\selectlanguage{ngerman}
	% Zeilenabstand
	\onehalfspacing
	% Formelnummerierung nur in Sections
	\numberwithin{equation}{section}
	% Anzahl sections tiefe
	\setcounter{secnumdepth}{3}
	% ToF tiefe
	\setcounter{tocdepth}{1}
	\setcounter{page}{-5}
	
	
	% Header & Footer definieren
	% Set header
	% []: Pages with chapter titles
	% {}: Pages without chapter titles
	% Inner head
	

	\ihead[\sffamily \flushleft \sffamily BAT Schlussbericht\\  \small \rightmark]{\sffamily \flushleft \sffamily BAT Schlussbericht\\  \small \rightmark}	
	%\rightmark takes first section on page as section in header
	%\leftmark takes last section on page as section in header
	% Center head
	\chead[]{}
	% Outer head
	\ohead[\sffamily \flushright \sffamily Hochschule Luzern\\  \small Technik \& Architektur]{\sffamily \flushright \sffamily Hochschule Luzern\\  \small Technik \& Architektur}
	
	% Set footer
	% Inner foot
	\ifoot[Daniel Zimmermann]{Daniel Zimmermann}
	% Center foot
	\cfoot[]{}
	% Outer foot
	\ofoot[\mark]{\pagemark}
	
	%Deckblatt
	\pagestyle{empty}
	\begin{titlepage}	
	\begin{center}	
		% Oberer Teil der Titelseite:
		
		\textsc{\bfseries \LARGE Bachelor-Thesis BAT }\\[1.0cm]
		
		\textsc{\Large Hochschule Luzern T\&A}\\[0.5cm]
		
			\textsc{\bfseries Studiengang}\\[0.2cm]
		
		\textsc{
			Elektrotechnik und Informationstechnologie}\\[0.2cm]
		
				\textsc{\bfseries Schwerpunkte}\\[0.2cm]
		
		\textsc{	Signalverarbeitung \& Kommunikation, \\
			Automation \& Embedded Systems}\\[0.5cm]
		
		% Title
		\newcommand{\HRule}{\rule{\linewidth}{0.5mm}}
		\HRule \\[1.0cm]
		{   \Huge \bfseries PIR Personendetektor\\
			\large \vspace{\baselineskip}
			
			\large Schlussbericht\\[1.0cm]
			
			\HRule \\[1.0cm]
			\large \vspace{\baselineskip}
		}
		\begingroup
		\parfillskip=0pt
		\large
		% text\hfill text
		
		\begin{minipage}[t]{0.48\textwidth}
			% Ich soll links stehen
			\raggedright						
			\emph{Autor:}\\
			Daniel Zimmermann\\
			daniel.zimmermann.01@stud.hslu.ch\\
			\hfill \break
			\\[4.4cm]
			% Unterer Teil der Seite
			{\large Klassifikation | Rücksprache\\ 
				Horw, 
				8. Juni 2018}
		\end{minipage}%
		\hfill
		\begin{minipage}[t]{0.48\textwidth}
			\raggedleft
			% und ich rechts
			\emph{Dozent:} \\
			Kilian Schuster \\
			kilian.schuster@hslu.ch\\[1.0cm]
			\emph{Industriepartner:} \\
			Hr. Markus Kappeler \\
			Schindler Aufzüge AG \\
			6030 Ebikon\\
			markus.kappeler@ch.schindler.com\\[1.0cm]
			\emph{Experte:}\\
			Erich Tschümperlin\\
			erich.tschuemperlin@bluewin.ch\\
		\end{minipage}%	
		\par\endgroup
		\hfill
		
	\end{center}	
\end{titlepage}

	\newpage
	
	%Seitenstyle
\pagestyle{scrheadings}	
\renewcommand{\chapterpagestyle}{scrheadings}
\renewcommand{\partpagestyle}{scrheadings}
	
	
	% Inhalt
	% Abstract
\newpage


\ihead[]{\sffamily \flushleft \sffamily BAT Schlussbericht\\  \small Eigenständigkeitserklärung}	
\ofoot[]{}

\input{tex/Selbstständigkeitskerlärung}
%Anderungshistorie
%Titel ohne Nummerierung
\chapter*{Abstract}
\label{chap:abstract_german}
Diese Dokumentation ist das Ergebnis der Bachelorarbeit an der Hochschule Luzern Technik \& Architektur für den Industriepartner Schindler Aufzüge AG. 


Im Rahmen der Arbeit soll geklärt werden, inwieweit sich bildgebende passiv Infrarot Sensoren für den Einsatzbereich in einem Personenaufzug eignen. Dafür steht State-of-the-Art Passiv Infrarot Sensor zur Verfügung. 


Thema und Zielsetzung: Stellen Sie zunächst Thema und Zielstellung der Arbeit vor.

Theorie: Vermitteln Sie Ihre Theorie(n) über das Thema und geben Sie an, auf was sich Ihre Theorie stützt.

Fragestellung: Teilen Sie mit, welche Fragen in der folgenden Arbeit beantwortet werden.

Quellen: Welche Quellen haben Sie für Ihre Arbeit genutzt bzw. wie haben Sie Ihre Frage(n) beantwortet?

Ergebnis: Führen Sie Ihre Ergebnisse auf, also teilen Sie mit, was Sie herausgefunden haben.

Fazit: Stellen Sie am Ende des Abstracts eine Quintessenz auf. Sie können Ihr Fazit auch mit einer Zukunftsprognose verbinden.


Zu Diagnosezwecken soll die Anwesenheit von Personen in Aufzugskabinen erfasst werden. Dazu bieten sich, unter anderem, Sensoren zur Erfassung der thermischen Strahlung an. Im Rahmen der Arbeit soll daher geklärt werden, inwieweit sich bildgebende PIR (passiv Infrarot) Sensoren dazu eignen. 


\chapter*{Abstract}
\label{chap:abstract_english}

This documentation is the result of a bachelor thesis at the Lucerne School
of Engineering and Architecture for the industry partner Schindler Aufzüge AG.

The task was the
realisation of a module, that maps the environment and creates a point cloud with the
measured data. The 3D-sensor Velodyne VLP-16 is available for this purpose.
The following chapters contain the experiences and results during the project from September
to December 2017. State-of-the-art projects have been investigated and compared.
After that, components and software for implementation were analysed and eunvaluated.
A total of three concepts were elaborated, which have different approaches. The first
concept turns the 3D sensor in a wide range of angle, while using servo motors. The
two other concepts are based on a endlessly rotating "tower". The idea behind it, are the
state-of-the-art projects. The difference between the two concepts is the position of the
signal processing unit. In the unrotated version, the unit is below in a static case. Only
the 3D-sensor is rotating for mapping. In the other version, the unit in the case is also
rotating. Only the interface is static.
The realised concept is similar to the unrotated version before. The realisation describes
the process, how the case and the electronic parts are assembled. In a separate topic, it
describes, how the Software with the Framework ROS is implemented and how it works
together with the hardware.
After the realisation the prototype was tested. Because of a

\ihead[]{\sffamily \flushleft \sffamily BAT Schlussbericht\\  \small Abstract}

%Inhaltsverzeichnis Dokumentation
\newpage
\tableofcontents
\ihead[]{\sffamily \flushleft \sffamily BAT Schlussbericht\\  \small \rightmark}
\newpage
\ofoot[]{}

%Zielbestimmung
\newpage
\chapter{Einleitung}
\label{chap:Einleitung}


\label{sec:Ausgangssituation}
Durch den technologischen Wandel, den die Industrie 4.0 sowie \ac{IoT}  mit sich bringen, entstehen in verschiedensten Einsatzbereichen neue Möglichkeiten. Da Sensoren zunehmend kleiner, vernetzter und günstiger werden, sowie stetig schnellere Prozessoren und größere Speicherkapazitäten zur Verfügung stehen, werden vermehrt auch in alltäglichen Situation intelligente Systeme eingesetzt. 

Gerade für Wartungs- und Diagnosezwecke von Personenaufzügen bieten solche intelligente Systeme ein bedeutendes Potential. Durch die ortsunabhängige Kommunikation von übergreifenden Netzwerken und der Echtzeitverarbeitung bieten solche Messeinheiten Alternativen zu teuren Servicegängen. Mittels ständiger Überwachung und Fernwartung können Probleme frühzeitig erkannt und behoben werden. Die Anforderungen an eine solche Messeinheit hängt jedoch stark von Einsatzort ab. Dabei spielen Langzeiteinsatz, Zuverlässigkeit, Flexibilität sowie auch der Energieverbrauch eine bedeutende Rolle.

Ein relevantes Messobjekt für eine solche Messeinheit ist die Anzahl Personen innerhalb eines Aufzugs. Da übliche Überwachungskameras und bildgebende TOF-Sensoren teuer sind und einen bedeutenden Energiebedarf besitzen, stellt sich in diesem Bereich die Frage nach einer Alternative.

\section{Aufgabenstellung}
\label{chap:Aufgabenstellung}

An diesem Punkt setzt nun die Aufgabenstellung dieser Bachelorarbeit an. Es soll die Eignung von bildgebenden Passiv Infrarot Sensoren (PIR) für eine solche Messeinheit geprüft werden.PIR in möglichst breiter und wegweisender Form beurteilt

 Es wird dabei der State-Of-the-Art Sensor AMG8834 von Panasonic verwendet. Mit diesem werden in einer ersten Phase grundlegende Grenzen und Eigenheiten dieses passiven Messprinzips erarbeitet. In einem weiteren Schritt wird auf der Grundlage von Messresultaten und Testdurchführungen ein Algorithmus entwickelt, mit welchem sich Personen innerhalb des Messbereichs detektieren lassen. Abschließend wird das Messprinzip in möglichst breiter und wegweisender Form beurteilt und eine Empfehlung für die Weiterführung gebildet werden. In erster Linie soll mit dieser Arbeit die Fragestellung geklärt werden, ob sich bilgebende Passiv Infrarot Sensoren für die Personendetektion eignen.

 
\section{Ziele}
\label{sec:Einleitung}
Ziel dieser Bachelorarbeit ist es, einen breiten und fundierten Katalog über die Möglichkeiten und Grenzen des PIR Sensors zu liefern. Auf der Basis der Testergebnisse wird ein Algorithmus zu erarbieten, mit dem sich Personen in einem Aufzug detektieren lassen. 

\section{Methodik}
\label{sec:Methodik}
Die gesamte Arbeit ist Etappenweise gegliedert. Dabei wiederholen sich Testphasen und Datenerfassungen und Auswertungen. Einzelne Testkonzepte geben Auskunft über den Projektablauf, sowie die durchgeführten Testspezifikationen. Das Projektmanagement in \todo{referenz} 

\chapter{Informationsbeschaffung}
\label{chap:Informationsbeschaffung}
Dieses Kapitel bietet fundamentale physikalische Gegebenheiten, sowie die relevanten Eigenheiten des verwendeten \ac{PIR}-Sensors. Da es sich um eine bildgebendes Messprinzip handelt, werden des Weiteren geometrische Aspekte erläutert. Schlussendlich bietet dieses Kapitel auch nötige Informationen über das Messobjekt bzw. die Messumgebung geliefert.

\section{Grid-Eye AMG8834}
\label{sec:AMG8834}

Der verwendete Panasonic AMG8834 ist ein bildgebender \ac{MEMS}-Sensor, der mit insgesamt 64 temperaturempfindlichen Thermosäulenelementen ausgestattet ist. Diese sind als 8x8 Pixel-Matrix auf den Chip aufgebracht. In Abbildung \ref{fig:Explosionsdarstellung} ist der Aufbau des Sensors dargestellt.
 
\begin{figure}[H]
	\centering
	\includegraphics[width=0.3\textwidth]
	{fig/grid_eye_aufbau.PNG}
	\caption[Schema des AMG8834 Sensors]{Schema des AMG8834 Sensors} \protect\cite{AMG8834}
	\label{fig:Explosionsdarstellung}
\end{figure}
Die eintreffenden Infrarotwellen werden durch die Siliziumlinse, welche einen \ac{FOV} von 60$^\circ$ besitzt, gefiltert. Dabei durchdringen lediglich langwellige Infrarotstrahlungen mit den Wellenlängen 8-13 $\mu$m die Linse. Dies entspricht dem dritten atmosphärischen Fenster.

In Abbildung \ref{fig:SchemaAMG8834} ist das Prinzipschema des Sensors darstellt. Das Messprinzip des Sensors wird im Unterkapitel \ref{subsec:seebeck} detailliert erläutert. Die entstandene Thermospannung wird durch die \ac{ASIC} des \ac{MEMS}-Sensor verarbeitet. Das selektierte Thermospannung wird verstärkt, mit dem integrierten Thermistor verglichen und mit dem \ac{ADC} gewandelt. Durch die hohe interne Verstärkung besitzt der Sensor jedoch bei normalen Bedingungen\footnote[2]{Umgebungstemperatur 0-80 $^\circ$C bei Luftfeuchtigkeit 15-85\%} eine Genauigkeit von +/- 3°C. 

\begin{figure}[H]
	\centering
	\includegraphics[width=0.75\textwidth]
	{fig/Circuit_AMG8834.PNG}
	\caption[Schema des AMG8834 Sensors]{Schema des AMG8834 Sensors} \protect\cite{AMG8834}
	\label{fig:SchemaAMG8834}
\end{figure}
 
Über die \ac{I2C} lassen sich die Werte der Thermoelemente und der Thermistoren je aus 2 Register auslesen. Die Messwerte werden alle 100 ms aktualisiert. Dabei werden lediglich 12 Bit pro Pixel für die Temperaturregister genutzt. Dies führt zu der kleinsten unterscheidbaren Größe von 0.25 $^\circ$C . Die Thermistorregister lassen sich mit der Auflösung von 0.625 $^\circ$C unterscheiden. In Abbildung \ref{fig:SchemaAMG8834} ist klar ersichtlich, dass die Umgebungstemperatur, bzw. die Temperatur, welche vom Thermistor gemessen wird direkten Einfluss auf die Pixelwerte besitzen. Variieren die Thermistorwerte aufgrund von Raumtemperaturschwankungen können die Pixelwerte dadurch bedeutende Abweichungen entstehen.

\section{Physikalische Aspekte}
\label{sec:Physik}
Dieser Abschnitt erläutert auf kurze und prägnante Weise, physikalischen Aspekte die dem Sensor zu Grunde liegen. Dies bietet die Grundlage für die Bestimmung der Störquellen und das Verhalten des Sensors bei entsprechenden äußeren Einwirkungen. Die Tabelle \ref{tab:Legende Physikalische Grössen} gibt die Bezeichnungen der nachfolgenden Formeln wieder.

\begin{table}[H]
	\centering
	\begin{tabular}{l|c|c}
		\rowcolor{gray} Grösse &  Bezeichnung  & Einheit \\
		\hline 
		Thermospannung &  $ U_{t}$ & $J$  \\ 
		\rowcolor{gray} Thermokraft P/N -Silizium  & $\alpha_{p},\alpha_{n}$ & $V/K$\\	
		Temperatur P/N -Silizium &  $T_{p},T_{n}$ & $V/K$ \\
		\rowcolor{gray}Wärmestrom &  $\dot{Q}$ & $J$  \\ 
		Emission & $\epsilon$ & $-$\\	
		\rowcolor{gray}Reflektion &  $\rho $ & $-$ \\
		Transmission & $\tau$ & $-$\\
		\rowcolor{gray}Absoprtion &  $\alpha$ & $-$  \\ 
		Strahlungsleistung & $\dot{Q}$ & $W$\\
		\rowcolor{gray}spektrale spezifische Ausstrahlung &  $M_{\lambda }$ & $W/sr$\footnote[3]{Steradiant: Messeinheit für den Raumwinkel} \\
		Planksches Wirkungsquantum &  $ h$ & Js \\ 
		\rowcolor{gray} Lichtgeschwindigkeit im Vakuum & $c $ & $ m/s$ \\ 
 		Stefan-Boltzmann-Konstante & $\sigma$ & $ W/m^2K^2 $ \\ 
	\end{tabular}
	\caption{Legende physikalische Grössen Konzeptzeichnungen}
	\label{tab:Legende Physikalische Grössen} 
\end{table} 


\subsection{Seebeck-Effekt}
\label{subsec:seebeck}
Die durch die konvexe Linse gesammelten Infrarotstrahlen verursachen auf den einzelnen Thermosäulenelemente (2), dass die Oberfläche erwärmt wird. Es entsteht zwischen der erwärmten, n-dotierten Siliziumschicht (4) und der kühleren p-dotierten Siliziumschicht (6) ein Temperaturgefälle.   

\begin{figure}[H]
	\centering
	\includegraphics[width=0.6\textwidth]
	{fig/Mems_Thermopile.PNG}
	\caption[Aufbau Thermosäulenelement]{Aufbau Thermosäule} \protect\cite{AMG8834}
	\label{fig:AufbauThermo}
\end{figure}

\todo{Kotrolle Thermosäule} Durch die unterschiedlichen Thermokräfte (auch Seebeckkoeffizienten) der zwei Halbleitermaterialien entsteht ein Potentialunterschied, den man an den Punkten 3 und 5 abgreifen kann. Diese Spannung $U_{t}$ ist die Grundlage des Messprinzips und wird mit Formel \ref{eq1} \protect\cite{AMG8834} beschrieben.


\begin{equation}
\label{eq1}
U_{t} = (\alpha_{p} + \alpha_{n})*(T_{p}+T_{n})
\end{equation}
\begin{center}

\end{center}
\myequations{Seebeck-Effekt}


\subsection{Strahlungstheorie}
\label{subsec:Strahlungstheorie}
Das vorherige Unterkapitel erläutert die Funktion des Sensors als Infrarotempfänger. Nicht unwesentlich ist weiter die Betrachtung des Senders. Grundsätzlich gilt, jeder Körper, der eine Temperatur oberhalb des absoluten Nullpunkt aufweist, strahlt Wärmestrahlung im Infrarotbereich ab. 

Im Allgemein wird für die Betrachtung vom Plank'schen Strahlungsgesetz ausgegangen. Nach dieser gilt für eine spektrale spezifische Ausstrahlung eines Schwarzkörpers mit der Temperatur T folgende Formel \protect\cite{Thermoformeln}:

\begin{equation}
\label{eq2}
M_{\lambda } = \frac{2\pi h c^2 }{\lambda^5}*\frac{1}{e^\frac{hc}{\lambda k_{B} T}-1}
\end{equation}
\myequations{Plank'sches Strahlungsgesetz}

Wie in der Formel ersichtlich ist die Ausstrahlung eines schwarzen Körper mit 5. Potenz von der Wellenlänge und exponentiell von der Temperatur abhängig. Durch die Siliziumlinse des Sensors werden somit Störquellen, welche andere Wellenlängen aufweisen, gefiltert. Dies ist vor allem bei Lichtquellen ein relevante Eigenschaft.

Das Stefan-Boltzmann-Gesetz \protect\cite{Thermoformeln} gibt die Strahlungsintensität $\dot{Q}$ eines idealen Temperaturstrahlers an. Diese Formel bietet für die Anwendung die relevantesten Erkenntnisse.


\begin{equation}
\label{eq3}
\dot{Q} = \frac{\mathrm{d} Q}{\mathrm{d} t} = \epsilon *\sigma * A * T_{obj}^4
\end{equation}
\myequations{Wärmestrahlung }

\begin{figure}[H]
	\centering
	\includegraphics[width=0.5\textwidth]
	{fig/seebeck2.PNG}
	\caption[Aufbau Thermosäule]{Aufbau Thermosäule} \protect\cite{seebeck}
	\label{fig:thermosäule}
\end{figure}

Diese Formel zeigt auf, dass die Wärmestrahlung eines Körpers im wesentlichsten (mit 4. Potenz)
von der eigenen Temperatur abhängig ist. Die Fläche A ist lediglich proportional. Dies verursacht, dass bereits flächenmäßig kleine, jedoch stark erwärmte Objekte im Messbereich einen bedeutenden Einfluss auf die Messresultate liefern können. Zusätzlich Verursachen Wärmequellen im nahen Umfeld des Messbereichs Abweichungen auf die Messresultate.

Des Weiteren weist diese Formel auf eine weitere Problematik dieses Messprinzips hin, der mit dem Emissionsgrad $\epsilon$  in Verbindung steht. Der Emissionsgrad $\epsilon$ ist ein materialabhängiger, jedoch wellenlängenunabhängier Faktor, welcher zwischen 0-1 angegeben wird. Dieser gilt für graue Körper d.h. für Körper, dessen Oberfläche auftreffende Strahlung nicht vollständig absorbieren. Diese Eigenheit gilt für alle realen Körper. 

Nach dem Energieerhaltungsgesetz \protect\cite{Thermoformeln} gilt für Transmission, Reflexion und Absorption die Formel \ref{eq4}.
\begin{equation}
\label{eq4}
\tau  + \alpha + \varphi  = 1
\end{equation}
\myequations{Schwarzer Stahler, Energieerhaltung}

Wobei bei thermischen Gleichgewicht angenommen werden kann, dass der Emissionsgrad der Absortion entpsricht.

\begin{equation}
\label{eq5}
\epsilon \approx  \alpha
\end{equation}
\myequations{Strahlung Energieerhaltung Festkörper}


Da es sich Aufzügen nur von Festkörper ausgegangen wird, fällt die Transmission $\tau$ aus der Gleichung. Es können lediglich Reflexionen oder die Emission des Körpers Einfluss auf die einwirkende Infrarotstrahlung nehmen. Weitere Betrachtungen diesbezüglich werden im Unterkapitel 2.4. gemach

\section{geometrische Aspekte}
\label{sec:geometrie}

Da die Strahlungsintensität mit zunehmender Distanz mit zweiter Potenz abnimmt, spielt die Distanz zum Messobjekt eine entscheidende Rolle. Ein weiteres Kriterium ist der begrenzten \ac{FOV} des Sensors von 60$^\circ$. In der nachstehenden Skizze (Abbildung \ref{fig:Geometrie}) sind die Verhältnisse perspektivisch dargestellt. Dabei wird von einer Raumhöhe von 2.10 m ausgegangen. (nach Standardkabine EN 81-70)   

\begin{figure}[H]
	\centering
	\includegraphics[width=0.5\textwidth]
	{fig/Humidity_Tolerance.PNG}
	\caption[Einfluss Luftfeuchtigkeit]{Einfluss Luftfeuchtigkeit} \protect\cite{AMG8834}
	\label{fig:Geometrie}
\end{figure}

Die räumliche Streckungen verursacht zusätzlich eine perspektivische Verzerrung, welcher in dieser Betrachtung nicht weiter beachtet wird. Zu sehen ist jedoch deutlich, dass bei der Messung von Personen die Messdistanz zwischen 10 bis 110 Zentimeter am relevantesten ist. In diesem Bereich kann jedoch mit dem aktuellen \ac{FOV} im besten Fall eine Fläche von 0.666 $ m/s^2 $ abgedeckt werden. Um eine Aufzugkabine mit 8 Personen\footnote[1]{Masse: (HxBxT) 2100 x 1100 x 1400 [mm]} bei mittlerem Messbereich wird im optimalen Fall ein Öffnungswinkel von 84°x 109°$^\circ$ benötigt. 

Problematisch kann in diesem Zusammenhang die Abschattung des Messbereichs durch grosse Personen sein, welche zentral positioniert sind. 

 

\section{Messobjekt und Messumgebung}
\label{sec:Messobjekt}
Dieses Kapitel beschreibt die Erkenntnisse bei der Betrachtung des Messobjekts und der Messumgebung. Dabei wurden einerseits die Kennwerte von Personen zusammengetragen, sowie die Messumgebung auf Störquellen und Einflussfaktoren begutachtet. Dank der Firma ARLEWO AG konnten unterschiedliche Aufzüge vermessen und bewertet werden. 

\subsection{Personen}
\label{subsec:Personen}

Die Reaktionen im menschlichen Körper sind auf eine Kerntemperatur von 37 °C. Am kältesten ist die Haut, die etwa 4 bis 7 Kelvin (Grad) kälter ist. Die Aufteilung der verschiedenen Arten der Wärmeabgabe beträgt bei einem ruhenden Menschen in einer Umgebung von 20 °C:


\begin{itemize}
	\item  46 \% Strahlung
	\item  33 \% Konvektion\footnote[2]{Konvektion bezeichnet die Wärmeabgabe an
		das umgebende Medium, in der Regel Luft}
	\item  19 \% Schwitzen
	\item   2 \% Atmung.
\end{itemize}

Die Höhe der Wärmeabgabe hängt im wesentlichen von der Schwere der Tätigkeit und von der Größe der Körperfläche ab. Daraus folgt, dass größere Personen mehr Wärme abgeben. Strahlung und Konvektion nehmen mit zunehmender Umgebungstemperatur bis zum Wert null bei 36 °C ab. Hat die Umgebung die Körpertemperatur erreicht, kann folglich durch Strahlung und Konvektion keine Wärme mehr abgeführt werden. In einer Umgebung mit Temperaturen oberhalb 37 °C kann also die Wärme nur noch durch Schwitzen abgeführt werden.\protect\cite{MenschWaerme}

\begin{figure}[H]
		\centering
	\begin{minipage}[b]{0.4\textwidth}
			\centering

\includegraphics[width=0.8\textwidth]{fig/person_waerme.JPG}
\label{fig:Waermebild}
\caption[Wäermebild eines \\Probanden]{Wärmebild eines \\Probanden}
	\end{minipage}
	\begin{minipage}[b]{0.2\textwidth}
\end{minipage}
	\begin{minipage}[b]{0.4\textwidth}
			\centering
\includegraphics[width=0.8\textwidth]{fig/person_waerme.JPG}
\label{fig:Waermebild2}
\caption[Wäermebild eines \\Probanden]{Wärmebild eines \\Probanden}

	\end{minipage}
	
\end{figure}

Ein weiterer Aspekt, der sehr stark ins Gewicht fällt, ist die Art der Bekleidung. In Abbildung \ref{fig:Waermebild} und Abbildung \ref{fig:Waermebild2} ist deutlich zu sehen, dass das thermische Profil einer Person durch die Bekleidung stark variiert. Bekleidungsfreie Zonen n 

\subsection{Personenaufzüge}

In diesem Unterkapitel wurde der Personenaufzug als Messobjekt näher betrachtet. Neben räumlichen Parametern wie Höhe, Grundfläche und Volumen spielen vor allem die Oberflächenbeschaffenheit bzw. das Oberflächenmaterial Rolle. Weitere thermische Einflussfaktoren finden sich in der Umgebungstemperatur und der verbauten Leuchtmittel.


Wie bereits in \ref{sec:geometrie}

In Abbildung \ref{fig:Edelstahlgewalzt} und \ref{fig:Edelstahlmatt} sind 
\begin{figure}[H]
	\centering
	\includegraphics[width=0.8\textwidth]
	{fig/Edelstahl_gewalzt.PNG}
	\caption[Edelstahl warmgewalzt]{Edelstahl warmgewalzt} \protect\cite{Edelstahl}
	\label{fig:Edelstahlgewalzt}
\end{figure}
\begin{figure}[H]
	\centering
	\includegraphics[width=0.8\textwidth]
	{fig/Edelstahl_matt.PNG}
	\caption[Edelstahl kaltgewalzt]{Edelstahl kaltgewalzt} \protect\cite{Edelstahl}
	\label{fig:Edelstahlmatt}	
\end{figure}




\begin{figure}[H]
	\centering
	\includegraphics[width=0.8\textwidth]
	{fig/Glas_bearbeitet.png}
	\caption[Emissionsgrad in Abhängikeit zur Wellenlnge]{Emissionsgrad in Abhängikeit zur Wellenlnge} \protect\cite{Edelstahl}
	\label{fig:Glas}	
\end{figure}

\section{Fazit}

Die Personenerkennung in Aufzügen mit \ac{PIR} Sensoren ist am meisten von der Individualität einer Person abhängig. Faktoren wie Körpertemperatur, Körpergröße und Bekleidung verursachen enorme Differenzen. Dadurch kann kein einheitliches Profil erstellt werden. Der geometrische Aspekte Weitere physikalische Gegebenheiten wie die Umgebungstemperatur oder indirekte Sonneneinstrahlung bewirken veränderte Bedingungen für den Messbereich, welche bei einer Messeinheit berücksichtigt werden müssen.     




\chapter{Testdurchführugen}
\label{chap:Testphasen}

Es wurden im Rahmen dieser Arbeit eine grosse Anzahl an Messungen und Testfällen durchgeführt. Die Testkonzepte im Anhang geben detailliert Auskunft über die Testdurchführung. Dieses Kapitel beschäftigt sich mit den bedeutendsten Ergebnissen.

\section{Grundlagenmessungen}

Um 



\section{Streuung}

\begin{figure}[H]
	\centering
	\includegraphics[width=0.5\textwidth]
	{fig/Distanz_140cm_std_.jpg}
	\caption[Streuung der einzelnen Pixel im Vergleich]{Streuung der einzelnen Pixel im Vergleich}
	\label{fig:Streuung}
\end{figure}


\section{Reflektion}




\section{Einfluss Störquellen}

Dieser Abschnitt befasst sich mit den Einfluss von externen Quellen auf den Sensor. Dabei spielen natürliche 



\section{Personenmessungen}
Bei der Personenmessungen wurden unterschiedliche Probanden in einem Aufzug ausgemessen auf dessen Wärmestrahlung.



\begin{figure}[H]
	\centering
	\includegraphics[width=0.8\textwidth, angle=270]
	{fig/Messraster.JPG}
	\caption[Personenmessung Messraster]{Personenmessung Messraster}
	\label{fig:Messraster}
\end{figure}













\section{Fazit}

\chapter{Personendetektion}
\label{chap:Personendetektion}


Dieser Abschnitt beschreibt das Vorgehen, um die Anzahl Personen in einem Aufzug zu erkennen. In einem ersten Schritt wird die Verarbeitung der Rohdaten aufgezeigt. Für den Auswertealgorithmus wurden mehrere unterschiedliche Aufzüge evaluiert und ein Profil erstellt. 



\begin{figure}[H]
	\centering
	\includegraphics[width=0.5\textwidth]
	{fig/person_175_shirt.jpg}
	\caption[Pixeldarstellung einer Person]{Pxiedarstellung einer Person}
	\label{fig:Pixelbild}
\end{figure}



\section{Datenverarbeitung}




\section{Datenmanipulation mittels Interpolation}

Die Auflösung von 8x8 Pixel bietet optisch nur begrenzte Aussagekraft über die Anzahl Personen in einem Aufzug. Daher wurde mittels MATLAB mehrere Interpolationsverfahren benutzt, um die Auflösung der Personenerkennung zu verbessern. Im Zusammenhang mit den Pixelwerten eignet sich eine bikubische



Da im Zusammenhang mit dem Auswertealgorithmus mittels TensorFlow




\subsection{Profilbildung}


Im Verlauf der Arbeit wurden mehrere 


\section{Symetrische Erweiterung}



\section{Musterauswertung}




\section{Aufbau neuronales Netzwerk}

Convolutional Networks work by moving small filters across the input image. This means the filters are re-used for recognizing patterns throughout the entire input image. This makes the Convolutional Networks much more powerful than Fully-Connected networks with the same number of variables. This in turn makes the Convolutional Networks faster to train




The convolutional filters are initially chosen at random, so the classification is done randomly. The error between the predicted and true class of the input image is measured as the so-called cross-entropy. The optimizer then automatically propagates this error back through the Convolutional Network using the chain-rule of differentiation and updates the filter-weights so as to improve the classification error. This is done iteratively thousands of times until the classification error is sufficiently low.


(W -F +2P) : S+ 1

\section{Convolution Neural Network}


\begin{figure}[H]
	\centering
	\includegraphics[width=1\textwidth]
	{fig/CCN.png}
	\caption[Aufbau des Convolutional Neural Network]{Aufbau des Convolutional Neural Network}
	\label{fig:CCN}
\end{figure}


Quelle: Computer Science and Engineering Department, University of Bridgeport, A Framework for Designing the Architectures of Deep Convolutional Neural Networks. Mai 2017, url:, http://www.mdpi.com/1099-4300/19/6?view=abstract&listby=pubdate_published+DESC%2Cfirstpage+DESC%2Cnumber+DESC&page_no=1 (besucht am 24.03. 2018).

\section{c}

Entweder alle Sekunde der Mittelwert auswerten, doer alle 100 ms die Daten auswerten 

\section{Fazit}



%Definition Umgebung
\newpage

\chapter{Empfehlung und Bewertung}
\label{Empfehlung_Vorgehen}

Dieses Kapitel beinhaltet eine Zusammenfassung der wichtigsten Erkenntnisse. Dabei werden die 

\section{Teilbewertung}


Bewertung von Auflösung
Bewertung von Geometrischen Aspekte
Bewertung Messprinzip

Bewertung Personenerkennung

Bewertung von Personenerkennung

\section{Gesamtbewertung}
\label{Fazit}


\section{Empfehlung}
\label{sec:Empfehlung}

Bewetungsraster


\section{Weiteres Vorgehen}

Es bietet sich an die Profilbildung mit diesem Messprinzip zu verfeinern, damit noch mehr Anwedungs
Weitere Profilbildung, 

Temperaturbereich erweitern

\section{Offene Punkte}

Dieser Abschnitt erläutert offene Punkte, welche im Rahmen der Arbeit nicht untersucht wurden.

\textbf{Bewegungfehlverhalten}\\
Bei der aktuellen Betrachtung wird weitgehend von still stehenden Personen ausgegangen und dies zeigt sich auch bei der Auswertung mit der Echtzeitmesseinheit. 


\textbf{thermische Grenzfälle}\\
Es konnten aufgrund fehlender Möglichkeiten keine Messungen durchgeführt werden, welche Grenzfälle abdeckten. Vor allem das Verhalten des Sensor bei Umgebungstemperatur von 0$^\circ$ und 37$^\circ$ bietet eventuell Erkenntnisse für den Anwendungsfall im Außenbereich.

\textbf{Bewegungfehlverhalten}\\

\textbf{Bewegungfehlverhalten}\\

\section{Ausblick}

Diese Bachelorarbeit hat sich mit dem dem Panasonic Grid-Eye AMG8834 befasst. Während der Informationsbeschaffung wurde dieser mit erhältlichen Sensoren anderer Hersteller verglichen und als State-Of-The-Art beurteilt.  
Ab Mai 2018 wurde von der Firma Melexis der Sensor MLX90640 auf den Markt gebracht. Dieser Sensor könnte die Lücken, welche der verwendete Sensor besitzt schließen. Der MLX90640 besitzt mit einer Auflösung von 24x32 Pixel bedeutende Darstellung- und Auswertemöglichkeiten. Der Einsatztemperatur erstreckt sich zwischen -40$^\circ$ bis 85$^\circ$, daher ist er auch für extremere Umgebungstemperaturen geeignet. Interessant ist bei diesem Sensor das Model mit dem Öffnungswinkel von 110$^\circ$x75$^\circ$. Der Öffnungswinkel könnte die geometrische Problematik aus Kapitel \ref{sec:geometrie} lösen und somit für den Einsatzbereich in Personenaufzügen besser geignet sein. Preislich ist dieser Sensor jedoch doppelt so teuer wie der AMG8834. Das entsprechende Datenblatt ist im digitalen Anhang \ref{AnhangE} angefügt. 




\input{tex/Fazit}

%Freigabe des Ind. Partners und Dozent
\newpage


%Inhalt Anhang
\appendix

\chapter{Meilensteinplan}
\label{AnhangA}

\chapter{Detaillierter Projektplan}
\label{AnhangB}

\chapter{Risikomanagement}
\label{AnhangC}

\chapter{Übersicht Datensätze }
\label{AnhangD}


\chapter{Digitale Projektanhänge}
\label{AnhangE}

Der Projektanhang enthält neben dem Schlussbericht und dem Projektmanagement, alle Skizzen, Rohdaten in strukturierer Form. Alle Matlab und Python-Codes sind entsprechend kommentiert und geben Auskunft über die erstellten Programme. Jeder Unterordner enthält ein "readme", welches zusätzliche Informationen enthält..
\section{Ordnerstruktur CD}

Die beiliegende CD hat folgende Ordnerstruktur.

\begin{enumerate}
	\item Abgabedokument
	\begin{itemize}
		\item Abgabedokument
	\end{itemize}
	\item Projektmanagement
	\begin{itemize}
		\item Aufgabenstellung
		\item Pflichtenheft
		\item Detaillierter Projektplan
		\item Risikomanagement
	\end{itemize}
	\item Graphiken
	\begin{itemize}
		\item Skizze Konzept Plattform
		\item Skizze Konzept Turm unrotierend
		\item Skizze Konzept Turm rotierend
	\end{itemize}
	\item Messdaten
	\begin{itemize}
	\item Testkonzepte \& Testmappen
	\end{itemize}
	\item Matlab Codes
	\begin{itemize}
		\item dxf-Files
		\item stl-Files
	\end{itemize}
	\item Tensorflow
	\begin{itemize}
		\item Laser\_3D
	\end{itemize}
	\item Datenblätter
	\begin{itemize}
		\item Velodyne VLP-16
		\item Einplatinencomputer
		\item Schrittmotor
		\item Schleifringe
	\end{itemize}
\end{enumerate}

\newpage





	
\end{document}

