%Titel ohne Nummerierung
\chapter*{Abstract}
\label{chap:abstract_german}
Diese Dokumentation ist das Ergebnis der Bachelorarbeit an der Hochschule Luzern Technik \& Architektur für den Industriepartner Schindler Aufzüge AG. 


Im Rahmen der Arbeit soll geklärt werden, inwieweit sich bildgebende passiv Infrarot Sensoren für den Einsatzbereich in einem Personenaufzug eignen. Dafür steht State-of-the-Art Passiv Infrarot Sensor zur Verfügung. 


Thema und Zielsetzung: Stellen Sie zunächst Thema und Zielstellung der Arbeit vor.

Theorie: Vermitteln Sie Ihre Theorie(n) über das Thema und geben Sie an, auf was sich Ihre Theorie stützt.

Fragestellung: Teilen Sie mit, welche Fragen in der folgenden Arbeit beantwortet werden.

Quellen: Welche Quellen haben Sie für Ihre Arbeit genutzt bzw. wie haben Sie Ihre Frage(n) beantwortet?

Ergebnis: Führen Sie Ihre Ergebnisse auf, also teilen Sie mit, was Sie herausgefunden haben.

Fazit: Stellen Sie am Ende des Abstracts eine Quintessenz auf. Sie können Ihr Fazit auch mit einer Zukunftsprognose verbinden.


Zu Diagnosezwecken soll die Anwesenheit von Personen in Aufzugskabinen erfasst werden. Dazu bieten sich, unter anderem, Sensoren zur Erfassung der thermischen Strahlung an. Im Rahmen der Arbeit soll daher geklärt werden, inwieweit sich bildgebende PIR (passiv Infrarot) Sensoren dazu eignen. 


\chapter*{Abstract}
\label{chap:abstract_english}

This documentation is the result of a bachelor thesis at the Lucerne School
of Engineering and Architecture for the industry partner Schindler Aufzüge AG.

The task was the
realisation of a module, that maps the environment and creates a point cloud with the
measured data. The 3D-sensor Velodyne VLP-16 is available for this purpose.
The following chapters contain the experiences and results during the project from September
to December 2017. State-of-the-art projects have been investigated and compared.
After that, components and software for implementation were analysed and eunvaluated.
A total of three concepts were elaborated, which have different approaches. The first
concept turns the 3D sensor in a wide range of angle, while using servo motors. The
two other concepts are based on a endlessly rotating "tower". The idea behind it, are the
state-of-the-art projects. The difference between the two concepts is the position of the
signal processing unit. In the unrotated version, the unit is below in a static case. Only
the 3D-sensor is rotating for mapping. In the other version, the unit in the case is also
rotating. Only the interface is static.
The realised concept is similar to the unrotated version before. The realisation describes
the process, how the case and the electronic parts are assembled. In a separate topic, it
describes, how the Software with the Framework ROS is implemented and how it works
together with the hardware.
After the realisation the prototype was tested. Because of a
