%Titel ohne Nummerierung
\chapter*{Abstract}
\label{chap:abstract_german}
This documentation is the result of the bachelor thesis PIR Person detector at the Lucerne School of Engineering and Architecture for the industry partner Schindler Aufzüge AG.

For maintenance and diagnostic purposes, the presence of persons in elevator cabins should be detected. Among other things, sensors for the detection of thermal radiation are suitable for this purpose. In the context of the work, it should therefore be clarified to what extent passive infrared imaging sensors (PIR) are suitable for use in a passenger elevator. 

A state-of-the-art PIR sensor is available for this purpose. The Panasonic Grid-Eye AMG8834 sensor offers only 8x8 pixels and measures the surface temperature in a limited field of view.  

In order to assess the suitability of the sensor, not only the physical and geometric properties are analysed, but also all sources of interference and influencing factors are determined. Several test procedures and measurement setups are used to identify and rate a wide variety of influencing factors. The analysis shows that the ambient temperature, the size of the person to be measured and the type of clothing play an important role in detecting people. Built-in light sources, reflections and emissions of the surrounding materials are determined as sources of interference.  

In a further step, a neural network is created using machine learning and previously prepared data sets, which reflects the quality of person recognition. The person recognition is only carried out with zero to four persons, as the sensor characteristics in the measuring range no longer permit. 

The suitability of passive infrared sensors in passenger elevators could be successfully verified with this work with corresponding restrictions, which are described in the document.
An evaluation master and corresponding recommendations offer the possibility for further investigations. 


\chapter*{Abstract}
\label{chap:abstract_german}
Diese Dokumentation ist das Ergebnis der Bachelorarbeit ac{PIR} Personendetektor an der Hochschule Luzern Technik \& Architektur für den Industriepartner Schindler Aufzüge AG. 

Für Wartungs- und Diagnosezwecke soll die Anwesenheit von Personen in Aufzugskabinen erfasst werden. Dazu bieten sich unter anderem Sensoren zur Erfassung der thermischen Strahlung an. Im Rahmen der Arbeit soll daher geklärt werden, inwieweit sich bildgebende passiv Infrarotsensoren (PIR) für den Einsatzbereich in einem Personenaufzug eignen. 

Dafür steht ein State-of-the-Art PIR-Sennsor zur Verfügung. Der verwendete Sensor Panasonic Grid-Eye AMG8834 bietet lediglich 8x8 Pixel und misst die Oberflächentemperatur in einem begrenzten Blickfeld.  

Um die Eignung des Sensors zu beurteilen werden neben der Analyse der physikalischen und geometrischen Eigenschaften, vor allem auch Störquellen und Einflussfaktoren ermittelt. In mehreren Testdurchführungen und Messaufbauten werden verschiedenste Einfußfaktoren identifiziert und beurteilt. Bei der Analyse stellt sich heraus, dass bei der Personenerkennung hauptsächlich die Umgebungstemperatur, die Größe der zu messenden Person, sowie die Bekleidungsform eine bedeutende Rolle spielen. Als Störquellen werden eingebaute Lichtquellen, sowie Reflexionen und Emissionen der umgebenden Materialien ermittelt.  

In einem weiteren Schritt wird mittels maschinellen Lernens und vorgängig vorbereiteten Datensätzen ein neuronales Netzwerk erstellt, welches die Qualität der Personenerkennung wiedergibt. Dabei wird die Personenerkennung lediglich mit null bis vier Personen durchgeführt, da die Sensoreigenschaften im Messbereich nicht mehr zulassen. 

Die Eignung von passiv Infrarot Sensoren in Personenaufzügen konnte mit dieser Arbeit unter entsprechenden Einschränkungen, welche im Dokument ausgeführt sind, erfolgreich verifiziert werden.
Ein Bewertungsraster und entsprechende Empfehlungen bieten die Möglichkeit für weitere Untersuchungen bzw. offenen Punkte.   



