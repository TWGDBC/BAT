\appendix

\chapter{Meilensteinplan}
\label{AnhangA}

\chapter{Detaillierter Projektplan}
\label{AnhangB}

\chapter{Risikomanagement}
\label{AnhangC}

\chapter{Übersicht Datensätze }
\label{AnhangD}

Die erstellten Datensätze wurden folgendermasse unterteilt:


\textbf{Profil 1}

Training-set:         126936 \\
- Test-set:             151936 \\
- Validation-set:       25000 \\


\textbf{Profil 2}

- Training-set:         126936 \\
- Test-set:             153181 \\
- Validation-set:       25000 \\

\textbf{Profil 3}
- Training-set:         126936 \\
- Test-set:             154461 \\ 
- Validation-set:       25000 \\


\chapter{Emissionsgradtabelle}
\label{AnhangE}





\chapter{Digitale Projektanhänge}
\label{AnhangDig}

Der Projektanhang enthält neben dem Schlussbericht und dem Projektmanagement, alle Skizzen, Rohdaten in strukturierer Form. Alle Matlab und Python-Programme sind entsprechend kommentiert und geben Auskunft über die Funktionen. Jeder Unterordner enthält ein "readme", welches zusätzliche Informationen enthält..
\section{Ordnerstruktur CD}


Die beiliegende CD hat folgende Ordnerstruktur.

\begin{enumerate}
	\item Abgabedokument
	\begin{itemize}
		\item BAT\_Schlussdokumentation
	\end{itemize}
	\item Projektmanagement
	\begin{itemize}
		\item Aufgabenstellung
		\item Meilensteinplan P2
		\item Detaillierter Projektplan Teil 1
		\item Detaillierter Projektplan Teil 2
		\item Risikomanagement
	\end{itemize}
	\item Testdurchführungen
	\begin{itemize}
		\item Testkonzepte \& Testmappen
		\item Matlab Messungen
	\end{itemize}
	\item Messdaten
	\begin{itemize}
	\item Testkonzepte \& Testmappen
	\end{itemize}
	\item TSoftware Personenerkennung
	\begin{itemize}
		\item 
		\item Datensätze
	\end{itemize}
	\item Tensorflow
	\begin{itemize}
		\item Laser\_3D
	\end{itemize}
	\item Datenblätter
	\begin{itemize}
		\item Panasonic AMG8834
		\item Melexis MLX90640
		\item Fluke
		\item Fluke 
	\end{itemize}
\end{enumerate}

\newpage



