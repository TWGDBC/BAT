\chapter{Einleitung}
\label{chap:Einleitung}


\label{sec:Ausgangssituation}
Durch den technologischen Wandel, den die Industrie 4.0 sowie \ac{IoT}  mit sich bringen, entstehen in verschiedensten Einsatzbereichen neue Möglichkeiten. Die Sensoren werden zunehmend kleiner, vernetzter und günstiger. Dazu stehen stetig schnellere Prozessoren und größere Speicherkapazitäten zur Verfügung, daher werden vermehrt auch in alltäglichen Situation intelligente Systeme eingesetzt. 

Für Wartungs- und Diagnosezwecke von Personenaufzügen bieten solche intelligente Systeme ein bedeutendes Potential. Durch die ortsunabhängige Kommunikation von übergreifenden Netzwerken und der Echtzeitverarbeitung bieten solche Messeinheiten Alternativen zu teuren Servicegängen. Mittels ständiger Überwachung und Fernwartung können Probleme frühzeitig erkannt und behoben werden. Die Anforderungen an eine solche Messeinheit hängt jedoch stark von Einsatzort ab. Dabei spielen Langzeiteinsatz, Zuverlässigkeit, Flexibilität, sowie auch der Energieverbrauch eine bedeutende Rolle.

Ein relevantes Messobjekt für eine solche Messeinheit ist unter anderem die Anzahl Personen innerhalb eines Aufzugs. Da übliche Überwachungskameras und bildgebende TOF-Sensoren teuer sind und einen bedeutenden Energiebedarf besitzen, stellt sich in diesem Bereich die Frage nach einer Alternative.

\section{Aufgabenstellung}
\label{chap:Aufgabenstellung}
An diesem Punkt setzt nun die Aufgabenstellung dieser Bachelorarbeit an. Es soll die Eignung von \ac{PIR} für eine solche Messeinheit geprüft werden. Dabei wird ein typischer bildgebender \ac{PIR }Sensor in möglichst breiter und wegweisender Form beurteilt. Es wird dabei der State-Of-the-Art Sensor AMG8834 von Panasonic verwendet. Mit diesem sollen in einer ersten Phase grundlegende Grenzen und Eigenheiten dieses passiven Messprinzips erarbeitet werden. In einem weiteren Schritt soll auf der Grundlage von Messresultaten und Testdurchführungen ein prototypische Messeinheit und einen Auswertealgorithmus entwickelt werden, mit welchem sich Personen innerhalb des Messbereichs detektieren lassen. Abschließend wird das Messprinzip beurteilt und eine Empfehlung für die Weiterführung gebildet. 

\section{Ziele}
\label{sec:Einleitung}
In erster Linie soll mit dieser Arbeit die Fragestellung geklärt werden, ob sich bildgebende \ac{PIR} für die Personendetektion in Personenaufzügen eignen. Ziel dieser Bachelorarbeit ist es, einen breiten und fundierten Katalog über die Möglichkeiten und Grenzen des Messprinzips zu liefern. Aus diesem Katalog wird eine Bewertung erstellt, welche auf Basis von Messungen und dessen Ergebnissen aufbaut. Diese Bachelorarbeit begrenzt sich auf die Analyse des Messprinzips von bildgebenden \ac{PIR} Sensoren. Es werden keine Vergleiche mit anderen Sensorarten und Messprinzipien durchgeführt.

\section {Methodik}
\label{sec:Methodik}
Das Vorgehen wurde anfänglich durch einen Meilensteinplan gegliedert und ist etappenweise aufgebaut. Als erstes wurde einen Zeitraum für die Informationsbeschaffung definiert. Danach wiederholen sich Testphasen, Datenerfassungen und Auswertungen. Einzelne Testkonzepte geben Auskunft über die Testdurchführungen, sowie die entsprechenden Testspezifikationen. 

Das Projektmanagement in Anhang \ref{AnhangA} - \ref{AnhangC} beinhaltet neben den detaillierten Projektplan auch die anfänglich definierten Meilensteine. Im detaillierten Projektplan sind neben den Tätigkeiten auch die zeitlichen Abschätzungen als Soll-/Ist-Vergleich angefügt. im Kapitel \ref{chap:Reflexion} Reflexion wird zum Projektmanagement kurz Stellung genommen und grössere Differenzen kommentiert.

Für die Datenverarbeitung und Aufbereitung wurde mittels Matlab und Python 3.5 programmiert. Für den Auswertealgorithmus wird das Prinzip des maschinellen Lernens angewendet. Dafür steht die Open-Source-Library Tensorflow r1.7 von Google zur Verfügung. Im Anhang \ref{AnhangD} sind die erarbeiteten Datensätze für Tensorflow kurz spezifiziert. Der gesamte programmierte Quellcode, die Rohdaten, sowie die vorbereiteten Datensätze stehen im digitalen Anhang \ref{AnhangE} zur Verfügung. 


