\chapter{Empfehlung und Bewertung}
\label{Empfehlung_Vorgehen}

Dieses Kapitel beinhaltet eine Zusammenfassung der wichtigsten Erkenntnisse. Dabei werden die 

\section{Teilbewertung}
\label{sec:Teilbewertung}

\subsection{Bewertung Auflösung}

Bewertung von Geometrischen Aspekte
Bewertung Messprinzip

Bewertung Personenerkennung

Bewertung von Personenerkennung

\section{Gesamtbewertung}
\label{sec:Gesamtbewetung}




\section{Empfehlung}
\label{sec:Empfehlung}



\section{Weiteres Vorgehen}

Es bietet sich an die Profilbildung mit diesem Messprinzip zu verfeinern, damit noch mehr Anwedungs
Weitere Profilbildung, 

Temperaturbereich erweitern

\section{Offene Punkte}

Dieser Abschnitt erläutert offene Punkte, welche im Rahmen der Arbeit nicht untersucht wurden.

\textbf{thermische Grenzfälle}\\
Es konnten aufgrund fehlender Möglichkeiten keine Messungen durchgeführt werden, welche Grenzfälle abdeckten. Vor allem das Verhalten des Sensor bei Umgebungstemperatur von 0$^\circ$ und 37$^\circ$ bietet eventuell Erkenntnisse für den Anwendungsfall im Außenbereich.

\textbf{Bewegungfehlverhalten}\\
Bei der aktuellen Betrachtung wird weitgehend von stillstehenden Personen ausgegangen und dies zeigt sich auch bei der Auswertung mit der Echtzeitmesseinheit. Bewegungen verursachen, dass sich das Wärmebild einer Person kurzzeitig zum Teil bedeutend verändert. Dies kann zu einer falschen Erkennung von der Personenanzahl führen. 



\textbf{Alternative Profile}\\
Im Rahmen der Arbeit wurden lediglich stehende erwachsene Personen mit Grössen zwischen 162 - 187 betrachtet. Es wurden im Rahmen der Arbeit keine Kinder, Tiere und rollstuhlgängige Personen ausgemessen. Dessen Wärmebilder können von den aktuellen Profilen abweichen und führen zu fehlerhaften Ergebnissen.

\textbf{Bewegungfehlverhalten}\\

\section{Ausblick}

Diese Bachelorarbeit hat sich mit dem dem Panasonic Grid-Eye AMG8834 befasst. Während der Informationsbeschaffung wurde dieser mit erhältlichen Sensoren anderer Hersteller verglichen und als State-Of-The-Art beurteilt.  
Ab Mai 2018 wurde von der Firma Melexis der Sensor MLX90640 auf den Markt gebracht. Dieser Sensor könnte die Lücken, welche der verwendete Sensor besitzt schließen. Der MLX90640 besitzt mit einer Auflösung von 24x32 Pixel bedeutende Darstellung- und Auswertemöglichkeiten. Der Einsatztemperatur erstreckt sich zwischen -40$^\circ$ bis 85$^\circ$, daher ist er auch für extremere Umgebungstemperaturen geeignet. Interessant ist bei diesem Sensor das Model mit dem Öffnungswinkel von 110$^\circ$x75$^\circ$. Der Öffnungswinkel könnte die geometrische Problematik aus Kapitel \ref{sec:geometrie} lösen und somit für den Einsatzbereich in Personenaufzügen besser geignet sein. Preislich ist dieser Sensor jedoch doppelt so teuer wie der AMG8834. Das entsprechende Datenblatt ist im digitalen Anhang \ref{AnhangE} angefügt. 




