\chapter{Empfehlung und Bewertung}
\label{Empfehlung_Vorgehen}

Dieses Kapitel beinhaltet eine Zusammenfassung der wichtigsten Erkenntnisse. Dabei werden die 


Bewertung von Auflösung
Bewertung von Geometrischen Aspekte
Bewertung Messprinzip

Bewertung von Personenerkennung

\section{Fazit}
\label{Fazit}


\section{Empfehlung}


\section{Weiteres Vorgehen}


\section{Ausblick}

Diese Bachelorarbeit hat sich mit dem dem Panasonic Grid-Eye AMG8834 befasst. Während der Informationsbeschaffung wurde dieser mit erhältlichen Sensoren anderer Hersteller verglichen und als State-Of-The-Art beurteilt.  
Ab Mai 2018 wurde von der Firma Melexis der Sensor MLX90640 auf den Markt gebracht. Dieser Sensor könnte die Lücken, welche der verwendete Sensor besitzt schließen. Der MLX90640 besitzt mit einer Auflösung von 24x32 Pixel bedeutende Darstellung- und Auswertemöglichkeiten. Der Einsatztemperatur erstreckt sich zwischen -40$^\circ$ bis 85$^\circ$, daher ist er auch für extremere Umgebungstemperaturen geeignet. Interessant ist bei diesem Sensor das Model mit dem Öffnungswinkel von 110$^\circ$x75$^\circ$. Der Öffnungswinkel könnte die geometrische Problematik aus Kapitel \ref{{sec:geometrie} lösen und somit für den Einsatzbereich in Personenaufzügen besser geignet sein. Preislich ist dieser Sensor jedoch doppelt so teuer wie der AMG8834. Das entsprechende Datenblatt ist im digitalen Anhang \label{AnhangE} angefügt. 



