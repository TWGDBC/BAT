\chapter{Informationsbeschaffung}
\label{chap:Informationsbeschaffung}
Dieses Kapitel bietet fundamentale physikalische Gegebenheiten, sowie die relevanten Eigenheiten des verwendeten \ac{PIR}-Sensors. Da es sich um eine bildgebendes Messprinzip handelt, werden des Weiteren geometrische Aspekte erläutert. Schlussendlich bietet dieses Kapitel auch nötige Informationen über das Messobjekt bzw. die Messumgebung geliefert.

\section{Grid-Eye AMG8834}

Der verwendete Panasonic AMG8834 ist ein bildgebender \ac{MEMS}-Sensor, der mit insgesamt 64 temperaturempfindlichen Thermosäulenelementen ausgestattet ist. Diese sind als 8x8 Pixel-Matrix auf den Chip aufgebracht. In Abbildung \ref{fig:Explosionsdarstellung} ist der Aufbau des Sensors dargestellt.
 
\begin{figure}[H]
	\centering
	\includegraphics[width=0.3\textwidth]
	{fig/grid_eye_aufbau.PNG}
	\caption[Schema des AMG8834 Sensors]{Schema des AMG8834 Sensors} \protect\cite{AMG8834}
	\label{fig:Explosionsdarstellung}
\end{figure}
Die eintreffenden Infrarotwellen werden durch die Siliziumlinse, welche einen \ac{FOV} von 60 $^\circ$ besitzt, gefiltert. Dabei durchdringen lediglich langwellige Infrarotstrahlungen mit den Wellenlängen 8-13 $\mu$m die Linse. Dies entspricht dem dritten atmosphärischen Fenster.

In Abbildung \ref{fig:SchemaAMG8834} ist das Prinzipschema des Sensors darstellt. Das Messprinzip des Sensors wird im Unterkapitel \ref{subsec:seebeck} detailliert erläutert. Die entstandene Thermospannung wird durch die \ac{ASIC} des \ac{MEMS}-Sensor verarbeitet. Das selektierte Thermospannung wird verstärkt, mit dem integrierten Thermistor verglichen und mit dem \ac{ADC} gewandelt. Durch die hohe interne Verstärkung besitzt der Sensor jedoch bei normalen Bedingungen\footnote[2]{Umgebungstemperatur 0-80 $^\circ$C bei Luftfeuchtigkeit 15-85\%} eine Genauigkeit von +/- 3°C. 

\begin{figure}[H]
	\centering
	\includegraphics[width=0.75\textwidth]
	{fig/Circuit_AMG8834.PNG}
	\caption[Schema des AMG8834 Sensors]{Schema des AMG8834 Sensors} \protect\cite{AMG8834}
	\label{fig:SchemaAMG8834}
\end{figure}
 
Über die \ac{I2C} lassen sich die Werte der Thermoelemente und der Thermistoren je aus 2 Register auslesen. Die Messwerte werden alle 100 ms aktualisiert. Es werden lediglich 12 Bit pro Pixel für die Temperaturregister genutzt. Dies führt zu der kleinsten unterscheidbaren Größe von 0.25 $^\circ$C . Die Thermistor-Register lassen sich mit der Auflösung von 0.625 $^\circ$C unterscheiden. 

\section{Physikalische Aspekte}
Dieser Abschnitt erläutert auf kurze und prägnante Weise, physikalischen Aspekte die dem Sensor zu Grunde liegen. Dies bietet die Grundlage für die Bestimmung der Störquellen und das Verhalten des Sensors bei entsprechenden äußeren Einwirkungen. Die Tabelle \ref{tab:Legende Physikalische Grössen} gibt die Bezeichnungen der nachfolgenden Formeln wieder.

\begin{table}[H]
	\centering
	\begin{tabular}{l|c|c}
		\rowcolor{gray} Grösse &  Bezeichnung  & Einheit \\
		\hline 
		Thermospannung &  $ U_{t}$ & $J$  \\ 
		\rowcolor{gray} Thermokraft P/N -Silizium  & $\alpha_{p},\alpha_{n}$ & $V/K$\\	
		Temperatur P/N -Silizium &  $T_{p},T_{n}$ & $V/K$ \\
		\rowcolor{gray}Wärmestrom &  $\dot{Q}$ & $J$  \\ 
		Emission & $\epsilon$ & $-$\\	
		\rowcolor{gray}Reflektion &  $\rho $ & $-$ \\
		Transmission & $\tau$ & $-$\\
		\rowcolor{gray}Absoprtion &  $\alpha$ & $-$  \\ 
		Strahlungsintensität & $\dot{Q}$ & $m/s$\\
		\rowcolor{gray}spektrale spezifische Ausstrahlung &  $M_{\lambda }$ & $m/s^2$ \\
		Planksches Wirkungsquantum &  $ h$ & Js \\ 
		\rowcolor{gray} Lichtgeschwindigkeit im Vakuum & $c $ & $ m/s$ \\ 
 		Stefan-Boltzmann-Konstante & $\sigma$ & $ rad/s^2 $ \\ 
	\end{tabular}
	\caption{Legende physikalische Grössen Konzeptzeichnungen}
	\label{tab:Legende Physikalische Grössen} 
\end{table} 


\subsection{Seebeck-Effekt}
\label{subsec:seebeck}
Die durch die konvexe Linse gesammelten Infrarotstrahlen verursachen auf den einzelnen Thermosäulenelemente (2), dass die Oberfläche erwärmt wird. Es entsteht zwischen der erwärmten, n-dotierten Siliziumschicht (4) und der kühleren p-dotierten Siliziumschicht (6) ein Temperaturgefälle.   

\begin{figure}[H]
	\centering
	\includegraphics[width=0.6\textwidth]
	{fig/Mems_Thermopile.PNG}
	\caption[Aufbau Thermosäulenelement]{Aufbau Thermosäule} \protect\cite{AMG8834}
	\label{fig:AufbauThermo}
\end{figure}

Durch die unterschiedlichen Thermokräfte (auch Seebeckkoeffizienten) der zwei Halbleitermaterialien entsteht ein Potentialunterschied, den man an den Punkten 3 und 5 abgreifen kann. Diese Spannung $U_{t}$ ist die Grundlage des Messprinzips und wird mit Formel \ref{eq1} \protect\cite{AMG8834} beschrieben.


\begin{equation}
\label{eq1}
U_{t} = (\alpha_{p} + \alpha_{n})*(T_{p}+T_{n})
\end{equation}
\begin{center}

\end{center}
\myequations{Seebeck-Effekt}


\subsection{Strahlungstheorie}

Das vorherige Unterkapitel erläutert die Funktion des Sensors als Infrarotempfänger. Nicht unwesentlich ist weiter die Betrachtung des Senders. Grundsätzlich gilt, jeder Körper, der eine Temperatur oberhalb des absoluten Nullpunkt aufweist, strahlt Wärmestrahlung im Infrarotbereich ab. 

Im Allgemein wird für die Betrachtung vom Plank'schen Strahlungsgesetz ausgegangen. Nach dieser gilt für eine spektrale spezifische Ausstrahlung eines Schwarzkörpers mit der Temperatur T folgende Formel \protect\cite{Thermoformeln}:

\begin{equation}
\label{eq2}
M_{\lambda } = \frac{2\pi h c^2 }{\lambda^5}*\frac{1}{e^\frac{hc}{\lambda k_{B} T}-1}
\end{equation}
\myequations{Plank'sches Strahlungsgesetz}

Wie in der Formel ersichtlich ist die Ausstrahlung eines schwarzen Körper mit 5. Potenz von der Wellenlänge und exponentiell von der Temperatur abhängig. Durch die Siliziumlinse des Sensors werden somit Störquellen, welche andere Wellenlängen aufweisen gefiltert. Dies ist vor allem bei Lichtquellen ein relevante Eigenschaft.

Das Stefan-Boltzmann-Gesetz \protect\cite{Thermoformeln} gibt die Strahlungsintensität Q eines idealen Temperaturstrahlers an und bietet für die Anwendung die relevantesten Erkenntnisse.

\begin{equation}
\label{eq3}
\dot{Q} = \frac{\mathrm{d} Q}{\mathrm{d} t} = \epsilon *\sigma * A * T_{obj}^4
\end{equation}
\myequations{Wärmestrahlung }

\begin{figure}[H]
	\centering
	\includegraphics[width=0.5\textwidth]
	{fig/seebeck2.PNG}
	\caption[Aufbau Thermosäule]{Aufbau Thermosäule} \protect\cite{seebeck}
	\label{fig:thermosäule}
\end{figure}

Diese Formel zeigt auf, dass die Wärmestrahlung eines Körpers im wesentlichsten (mit 4. Potenz)
von der eigenen Temperatur abhängig ist. Die Fläche A ist lediglich proportional. 

Der Emissionsgrad $\epsilon$ ist ein materialabhängiger, jedoch wellenlängenunabhängier Faktor, welcher zwischen 0-1 angegeben wird. Dieser gilt für graue Körper d.h. für Körper, dessen Oberfläche auftreffende Strahlung nicht vollständig absorbiert. Diese Eigenheit gilt für alle realen Körper. Bei thermischen Gleichgewicht kann zusätzlich davon ausgegangen werden, dass die Emission dem Absorptionswert entspricht.

\begin{equation}
\label{eq4}
\epsilon = \alpha
\end{equation}
\myequations{Strahlung Energieerhaltung Festkörper}

Nach dem Energieerhaltungsgesetz \protect\cite{Thermoformeln} gilt für die 
\begin{equation}
\label{eq5}
\tau  + \alpha + \varphi  = 1
\end{equation}
\myequations{Schwarzer Stahler, Energieerhaltung}

Da in Aufzügen nur von Festkörper ausgegangen wird, fällt die Transmission $\tau$ aus der Gleichung. Es können lediglich Reflexionen oder die Emission des Körpers Einfluss auf die einwirkende Infrarotstrahlung nehmen.

Weitere Betrachtungen werden im Unterkapitel 2.4. diesbezüglich gemacht.

\section{geometrische Aspekte}
\label{sec:geometrie}

Da die Strahlungsintensität mit zunehmender Distanz mit zweiter Potenz abnimmt, spielt die Distanz zum Messobjekt eine entscheidende Rolle. Ein weiteres Kriterium ist der begrenzten \ac{FOV} des Sensors von 60$^\circ$ In der nachstehenden Skizze (Abbildung \ref{fig:Geometrie}) sind die Verhältnisse perspektivisch dargestellt. Dabei wird von einer Raumhöhe von 2.10 m ausgegangen. (nach Standardkabine EN 81-70)   

\begin{figure}[H]
	\centering
	\includegraphics[width=0.5\textwidth]
	{fig/Humidity_Tolerance.PNG}
	\caption[Einfluss Luftfeuchtigkeit]{Einfluss Luftfeuchtigkeit} \protect\cite{AMG8834}
	\label{fig:Geometrie}
\end{figure}

Die räumliche Streckungen verursacht zusätzlich eine perspektivische Verzerrung, welcher in dieser Betrachtung nicht weiter beachtet wird. Zu sehen ist jedoch deutlich, dass bei der Messung von Personen die Messdistanz zwischen 10 bis 110 Zentimeter am relevantesten ist. In diesem Bereich kann jedoch mit dem aktuellen \ac{FOV} im besten Fall eine Fläche von 0.666 $ m/s^2 $ abgedeckt werden. Um eine Aufzugkabine mit 8 Personen\footnote[1]{Masse: (HxBxT) 2100 x 1100 x 400 [mm]} mit entsprechenden Messdistanzen wird ein Öffnungswinkel von -XX$^\circ$ benötigt. 

Problematisch kann in diesem Zusammenhang die Abschattung des Messbereichs durch grosse Personen sein, welche zentral positioniert sind. 

 

\section{Messobjekt und Messumgebung}

Dieses Kapitel beschreibt die Erkenntnisse bei der Betrachtung des Messobjekts und der Messumgebung. Dabei wurden einerseits die Kennwerte von Personen zusammengetragen, sowie die Messumgebung auf Störquellen und Einflussfaktoren begutachtet. Dank der Firma ARLEWO AG konnten unterschiedliche Aufzüge vermessen und bewertet werden. 

\subsection{Personen}

Die Reaktionen im menschlichen Körper sind auf eine Kerntemperatur von 37 °C eingestellt mit einer Toleranz von etwa + 0,5 Kelvin (Grad). Am kältesten ist die Haut, die etwa 4 bis 7 Kelvin (Grad) kälter ist. Die Aufteilung der verschiedenen Arten der Wärmeabgabe beträgt bei einem ruhenden Menschen in einer Umgebung von 20 °C:

\begin{enumerate}
\item 
\begin{itemize}
	\item  46 \% Strahlung
	\item  33 \% Konvektion
	\item  19 \% Schwitzen
	\item   2 \% Atmung.
\end{itemize}
\end{enumerate}	

Die Höhe der Wärmeabgabe hängt im wesentlichen von der Schwere der Tätigkeit und von der Größe der Körperfläche ab. Daraus folgt, dass größere Personen mehr Wärme abgeben. Diese Art der Wärmeabgabe nimmt mit der Umgebungstemperatur bis zum Wert null bei 36 °C ab. Hat die Umgebung nämlich die Körpertemperatur erreicht, kann folglich durch Strahlung und Konvektion keine Wärme mehr abgeführt werden.\protect\cite{MenschWaerme}

In einer Umgebung mit Temperaturen oberhalb 37 °C kann also die Wärme
nur noch durch Schwitzen abgeführt werden. Bei mittelschwerer Arbeit verdoppelt sich
ungefähr die Wärmeabgabe des Menschen gegenüber dem ruhigen Sitzen, da die Muskeln,
wie bereits erwähnt, zu 80 \% Abwärme erzeugen. 

\begin{figure}[H]
		\centering
	\begin{minipage}[b]{0.4\textwidth}
			\centering
\label{fig:Waermebild}
\includegraphics[width=0.8\textwidth]{fig/person_waerme.JPG}
\caption[Wäermebild eines \\Probanden]{Wärmebild eines \\Probanden}
	\end{minipage}
	\begin{minipage}[b]{0.2\textwidth}
\end{minipage}
	\begin{minipage}[b]{0.4\textwidth}
			\centering
\includegraphics[width=0.8\textwidth]{fig/person_waerme.JPG}
\caption[Wäermebild eines \\Probanden]{Wärmebild eines \\Probanden}
\label{fig:Waermebild2}
	\end{minipage}
	
\end{figure}


\subsection{Personenaufzüge}

In diesem Unterkapitel wurde das Messobjekt "Personenaufzug" näher betrachtet. Neben räumlichen Parametern wie Höhe, Grundfläche und Volumen spielen vor allem die Oberflächenbeschaffenheit bzw. das Oberflächenmaterial eine wichtige Rolle. Weitere thermische Einflussfaktoren finden sich in der Umgebungstemperatur und der eingebauten Leuchtmittel.


In Abbildung \ref{fig:Edelstahlgewalzt} und \ref{fig:Edelstahlmatt} sind 
\begin{figure}[H]
	\centering
	\includegraphics[width=0.8\textwidth]
	{fig/Edelstahl_gewalzt.PNG}
	\caption[Schema des AMG8834 Sensors]{Schema des AMG8834 Sensors} \protect\cite{Edelstahl}
	\label{fig:Edelstahlgewalzt}
\end{figure}
\begin{figure}[H]
	\centering
	\includegraphics[width=0.8\textwidth]
	{fig/Edelstahl_matt.PNG}
	\caption[Schema des AMG8834 Sensors]{Schema des AMG8834 Sensors} \protect\cite{Edelstahl}
	\label{fig:Edelstahlmatt}	
\end{figure}











\section{verwendete Software}



\section{Fazit}


\begin{figure}[H]
	\centering
	\includegraphics[width=0.6\textwidth]
	{fig/accuracy.PNG}
	\caption[Messgenauigkeit]{Messgenauigkeit} \protect\cite{AMG8834}
	\label{fig:Temperaturbereich}
\end{figure}
