\chapter{Personendetektion}
\label{chap:Personendetektion}



\begin{figure}[H]
	\centering
	\includegraphics[width=0.5\textwidth]
	{fig/person_175_shirt.jpg}
	\caption[Pixeldarstellung einer Person]{Pxiedarstellung einer Person}
	\label{fig:Pixelbild}
\end{figure}


\section{Datenverarbeitung}

\subsection{Profilierung}

\Cross Compiling Tensorflow



\section{Musterauswertung}



\section{Interpolation}

Die Auflösung von 8x8 Pixel bietet nur begrenzte Aussagekraft. Daher wurde mittels MATLAB mehrere Interpolationsverfahren durchgeführt um die Auflösung der Personenerkennung zu verbessern.





\section{Aufbau neuronales Netzwerk}

Convolutional Networks work by moving small filters across the input image. This means the filters are re-used for recognizing patterns throughout the entire input image. This makes the Convolutional Networks much more powerful than Fully-Connected networks with the same number of variables. This in turn makes the Convolutional Networks faster to train




The convolutional filters are initially chosen at random, so the classification is done randomly. The error between the predicted and true class of the input image is measured as the so-called cross-entropy. The optimizer then automatically propagates this error back through the Convolutional Network using the chain-rule of differentiation and updates the filter-weights so as to improve the classification error. This is done iteratively thousands of times until the classification error is sufficiently low.


(W -F +2P) : S+ 1

\section{convolution}


\section{c}

Entweder alle Sekunde der Mittelwert auswerten, doer alle 100 ms die Daten auswerten 

\section{Fazit}


\section{Symetrische Erweiterung}