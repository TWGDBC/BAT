\chapter{Reflexion}
\label{chap:Reflexion}

Dieses Kapitel beinhaltet neben den bedeutendsten Erläuterungen zum Projektmanagement auch ein persönliches Resümee im Schlusswort. Mit den entsprechenden Danksagung an alle Personen, welche mich bei dieser Arbeit unterstützt haben schliesst diese Dokumentation..


\section{Erläuterungen zum Projektmanagement}
\label{Projektmanagement}
Im Rahmen dieser Arbeit wurden anfänglich die Meilensteine definiert und ein Risikomanagement erstellt. Da diverse Meilensteine voneinander abhängig sind, mussten im Verlauf der Arbeit diese angepasst werden. In Anhang \ref{AnhangA} und \ref{AnhangC} ist der Meilensteinplan V3 und das Risikomanagement angefügt.

Neben der Planung wurden die wöchentlichen Besprechungen protokolliert und entsprechende Erkenntnisse in die Planung miteinbezogen. 

Der detaillierte Projektplan V3 im Anhang \ref{AnhangB} bietet vollständigen Überblick über die erledigten Tätigkeiten während der gesamten Arbeit. Dabei sind die Meilensteine \textcolor{darkgray}{grau} markiert.

Es wurde ein SCRUM ähnliches Verfahren angewendet, um alle anstehenden Aufgaben systematisch abzuarbeiten. Der detaillierte Projektplan gibt zudem auch den geschätzten und effektiven Zeitaufwand der Tätigkeit wieder.

Der gesamthaft, berechnete Zeitaufwand ist in der Tabelle  \ref{tab:Zeitaufwand} dargestellt. Die entstandenen Differenz des Soll-Ist-Vergleich wird in nachfolgenden Abschnitt erläutert.

\begin{table}[H]
	\centering
	\caption[Soll-Ist-Vergleich zeitlicher Aufwendungen in {[h]}]{Soll-Ist-Vergleich zeitlicher Aufwendungen in [h]}
	\begin{tabular}{|c|c|c|}
		\hline
		\rowcolor[HTML]{9B9B9B} 
	\multicolumn{1}{|c|}{\cellcolor[HTML]{9B9B9B}\textbf{Soll-Aufwand}} &  \multicolumn{1}{c|}{\cellcolor[HTML]{9B9B9B}{\color[HTML]{333333} \textbf{Ist-Aufwand}}} & \textbf{zeitliche Differenz} \\ \hline
		405.5
		& 457.25                                                                                           &    
		52.25                \\ \hline
	\end{tabular}
	\label{tab:Zeitaufwand}
\end{table}

Im allgemeinen wurde die Aufwände der einzelnen Tätigkeiten eher unterschätzt. Dabei sind vor allem die Zeitdifferenzen der Projektphasen; Hardware/Software V1, Datenerfassung \& Auswertung TP1 und die Dokumentation ausschlaggebend für die Differenz.

Für die Informationsbeschaffung musste mehr Zeit aufgewendet werden, bis die Grundlagen des Messprinzips verstanden wurden. Daher wurde der Zeitraum verlängert.

Der Testaufbau und die Durchführungen benötigen enorm viel Zeit, da jede Messung vor- und nachbearbeitet wurde. Dabei nehmen die Parameteridentifikation und die Auswertung der Messergebnisse zusätzliche Zeit in Anspruch. 

Die Softwareimplementierung des \ac{CNN} war bedeutend aufwendiger als angenommen. Viele kleine Fehler verursachten Verzögerungen, daher musste dieser Projektabschnitt um eine Woche verlängert werden. 

Es wurde noch Zeit aufgewendet eine Echtzeitmesseinheit zu erstellen. Diese Messeinheit war in der anfänglichen Planung nicht enthalten. 

Ansonsten konnten die geschätzten zeitlichen Rahmen eingehalten werden. Für die gesamte Projektplanung sowie der Aktualisierungen des Zeitplans und der Tätigkeiten wurde eine aufsummierte Aufwendung von 15 Stunden 



\newpage
\section{Schlusswort}
\label{sec:Schlusswort}
Mit der zunehmenden Verbreitung von Internet of Things in alltäglichen Situation bieten simple Sensoren neue Anwendungsmöglichkeiten. Das Potential von neuronalen Netzwerken für die Bilderkennung zeigen Vorzeigeprojekte wie das \acsfont{MNIST Dataset}. Auch in der thermografischen Bilderkennung bietet dieses Modellierungsverfahren interessante Ansätze.

Da keine persönlichen Grundkenntnisse in diesem Themengebiet vorgängig vorhanden war, war der Einstieg in dieses Thema anfänglich zeitraubend. Dennoch konnte durch gute Literaturen und Unterstützung an der Hochschule Luzern entsprechende Kenntnisse erworben werden. 

Diese Bachelorarbeit bietet ein breites Spektrum an Know-How um die vorgegebene Aufgabenstellung zu lösen. Dabei sind neben den schulischen Kenntnissen, vor allem auch Projekthandling und Flexibilität gefragt. Ein erwähnenswerter Punkt an dieser Stelle ist vor allem auch die Schwierigkeit mit diesen enormen Datenmengen umzugehen. Es benötigt viel Zeit und Fleiss die Datensätze akribisch zu kontrollieren und Unstimmigkeiten zu korrigieren. Bereits einige fehlerhafte Datenframes führen zu  falschen Ergebnissen. Da die Bewertung auf diesen Daten beruht, hängt die Qualität von der Richtigkeit der Datensätze ab.

Abschließend ist noch zu erwähnen, dass gerade der Bereich der intelligenten Systemen zukünftig ein mehr und mehr zentrales Thema für angehende Elektroingenieure wird. Daher ist es erfreulich, dass gerade solche Aufgaben in Zusammenarbeit mit Industriepartnern an der Hochschule Luzern in Angriff genommen werden. Besten Dank dafür.



  



\section{Danksagung}

An dieser Stelle möchte ich mich bei allen bedanken, die mich bei der Ausführung dieser Arbeit unterstützt haben. 
Zuallererst gebührt der Dank an Kilian Schuster, der mich als betreuender Dozent bei dieser Bachelorarbeit tatkräftige unterstützt hat, sowie mit wertvollen Hinweisen und ehrlichen Rückmeldungen zur Seite gestanden ist. Mein Dank geht auch an Manuel Serquet, der mich mt TensorFlow vertraut gemacht hat und diverse Unklarheiten klären konnte. 

Ebenfalls bedanken ich mich bei den Gegenlesern Julia Schuler und Marie-Theres Zimmermann für die syntaktische und inhaltliche Korrektur der wissenschaftlichen Dokumentation.

Ein speziellen Dank geht an die Immobilienverwaltungsfirma ARLEWO in Stans, welche mir ein breites Spektrum an Schindler Aufzügen bereitstellte, damit die Feldmessungen praxisnahe durchgeführt werden konnten. An diesem Punkt besten Dank auch allen Probanden, welche sich für die Feldmessungen zur Verfügung gestellt haben.