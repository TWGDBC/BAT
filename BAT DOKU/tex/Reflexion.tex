\chapter{Reflexion}
\label{chap:Reflexion}

Dieses Kapitel beinhaltet neben den bedeutendsten Erläuterungen zum Projektmanagement auch ein persönliches Resümee im Schlusswort. Mit den entsprechenden Danksagung an alle Personen, welche mich bei dieser Arbeit unterstützt haben, endet der inhaltliche Teil.


\section{Erläuterungen zum Projektmanagement}
Im Rahmen dieser Arbeit wurde anfänglich die Meilensteine definiert, welche im Anhang \ref{AnhangA} zu sehen sind. Der detaillierte Projektplan im Anhang \ref{AnhangB} bietet vollständigen Überblick über die erledigten Tätigkeiten. 


DiProblematiken nach dem Risikomanagement in \ref{AnhangC} angegangen.

In Anhang \ref{AnhangA}


\section{Schlusswort}

Mit der zunehmenden Verbreitung von Internet of Things in alltäglichen Situation bieten simple Sensoren neue Anwendungsmöglichkeiten. Das Potential von neuronalen Netzwerken für die Bilderkennung zeigen Vorzeigeprojekte wie das \acsfont{MNIST Dataset}. Auch in der thermografischen Bilderkennung bietet dieses Modellierungsverfahren Ansätze. 

In Aufzügen kann die Thermographie . Die Schwierigkeit liegt in der Individualität einer Person, welche grosse Unterschiede im Wärmebild einer Person verursacht.   



\section{Danksagung}

An dieser Stelle möchte ich mich bei allen bedanken, die mich bei der Ausführung dieser
Arbeit unterstützt haben. 
Zuallererst gebührt der Dank an Kilian Schuster, der mich als betreuender Dozent bei dieser Bachelorarbeit tatkräftige unterstützt hat, sowie mit wertvollen Hinweisen und ehrlichen Rückmeldungen zur Seite gestanden ist. Mein Dank geht auch an Manuel Serquet, der mich mt TensorFlow vertraut gemacht hat und einige Unklarheiten klären konnte. 

Ebenfalls bedanken ich mich bei den Gegenlesern Julia Schuler und Marie-Theres Zimmermann für die syntaktische und inhaltliche Korrektur der wissenschaftlichen Dokumentation.

Ein speziellen Dank geht an die Immobilienverwaltungsfirma ARLEWO in Stans, welche mir ein breites Spektrum an Schindler Aufzügen bereitstellte, damit die Feldmessungen praxisnahe durchgeführt werden konnten. An diesem Punkt besten Dank auch allen Probanden, welche sich für di Feldmessungen zur Verfügung gestellt haben.