\begin{acronym}[LAMS aaaaaaaaa]


\acro{ADC}{Analog/Digital-Converter} \\
\acroextra{Element zur Umsetzung von analogen Signalen}

\acro{CNN}{Convolutional Neural Network} \\
\acroextra{Künstliches neuronales Netzwerk, vorwiegend für Bildverarbeitung}

\acro{ASCII}{American Standard Code for Information Interchange} \\
\acroextra{Standardisierte 7-Bit Zeichencodierung}
\acro{ASIC}{Anwendungsspezifische Integrierte Schaltung} \\
\acroextra{eine elektronische Schaltung, die als integrierter Schaltkreis realisiert wurde}

\acro{CSV}{Comma-separated values}  \\
\acroextra{Simples Dateiformat, welches Daten kommasepariert anlegt}


\acro{FOV}{Field Of View}             	\\ 
\acroextra{Bezeichnet den Bereich im Bildwinkel eines optischen Sensors}


\acro{IoT}{Internet of Things} \\
\acroextra{Sammelbegriff für vernetzte, elektronische Systeme}

\acro{I2C}{Inter-Integrated Circuit}  \\
\acroextra{Serieller Datenbus für asynchrone Datenübertragung}

\acro{MEMS}{Mikroelektromechanisches System} 	\\ 
\acroextra{Miniaturisiertes System in der Grössenordnung von Mikrometern mit eigener Logik}

\acro{MNIST Dataset}{
	Modified National Institute of Standards and Technology Dataset} 	\\ 
\acroextra{Bekannter Datensatz von handgeschriebenen Ziffern zum Gebrauch als Trainingsset in der Anwendung von maschinellen Lernens}


\acro{NETD}{Rauschäquivalente Temperaturdifferenz} \\
\acroextra{Ein Maß für das Bildrauschen einer Infrarotkamera}

\acro{PCB}{Leiterplatte}
\acroextra{Mehrschichtige Träger für elektronische Bauteile und Verbindungen}

\acro{PIR}{Passiv Infrarotsensoren}	\\ 
\acroextra{Sensor der auf langwellige Infrarotstrahlen reagiert}

\acro{RPI3}{Raspberry Pi 3}\\
\acroextra{Kompakter Einplatinencomputer}

\acro{UART}{Universal Asynchronous Receiver Transmitter} \\
\acroextra{Schnittstelle zur asynchronen seriellen Datenübertragung}


\end{acronym} 
