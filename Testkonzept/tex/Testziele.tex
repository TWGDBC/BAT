%Titel ohne Nummerierung
\chapter{Testziele}

Das vorliegende Testkonzept besitzt vorwiegend das Ziel möglichst in einer breiter und wegweisender Form den Einsatz von passiv Infrarot Sensoren (PIR) für die Personendetektion zu beurteilen. 

Mittels diversen Testdurchführungen sollen in einer ersten Phase die physikalischen Grenzen im Allgemeinen, sowie im dafür vorgesehenen Einsatzbereich untersucht werden. 

Beim untersuchten Einsatzbereich handelt es sich hauptsächlich um Personenaufzüge, daher werden neben den allgemeinen Messungen auch konkrete Messungen an ausgewählten Objekt durchgeführt.  

In einer zweiten Phase steht die Personendetektion im Vordergrund. Dabei wird der erstellte Algorithmus zur Personendetektion auf seine Funktion getestet. Dabei werden bei verschiedenen Bedingungen und Testumgebungen vordefinierte Muster auf Vollständigkeit geprüft.
